\chapter{Methodology}  \label{sec:method}
%Goal of this chapter
%How DEA build upon GMA
%Main steps and decisions taken

\section{Modifications to the \ac{gma}} \label{sec:modifications}
%How GMA serves as basis
%Which parts are reused and which need modification
%Describe therefore the starting point of the DEA

\section{Design and Implementation of the \ac{dea}} \label{sec:designandimplementation}

\subsection{Problem Definition} \label{sec:problemdefinition}
%Main challenge: extending re-identifications beyond
%Reconstructing deterministic relationship between used encodings scheme (which are using hash functions) and plaintext
%Why GMA's fail to do this

\section{Data Representation} \label{sec:representation}
%Describe structure of encoded records (bf, tmh, tsh) and their representations
%Explain how re-identified individuals serve as labeled training data
%Discuss preprocessing steps required for effective neural network training

\subsection{\ac{ann} Architecture for \ac{dea}} \label{sec:architecture}
%Justify use of neural networks to predict missing information
%Present architecture for each encoding scheme: Input, Hidden and output layer
%Explain choice of activation functions, loss functions and optimizer
%How to handle variable length inputs (padding, truncation) for TMH

\subsection{Training the Model} \label{sec:training}
%Detail dataset split (training, validation, test)
%Discuss loss cumputation & performance metrics
%Explain training process and hyperparameter tuning
