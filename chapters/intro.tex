% !TeX root = main.tex
%%%%%%%%%% SVN Info %%%%%%%%%%%%%%%%%%%%%%%%%%%%%%%%%%%%%%%%
%% $Date: 2020-02-05 16:17:49 +0100 (Mi, 05 Feb 2020) $
%% $Author: cmueller $
%% $Rev: 5 $
%%%%%%%%%%%%%%%%%%%%%%%%%%%%%%%%%%%%%%%%%%%%%%%%%%%%%%%%%%%%

\chapter{Introduction}  \label{sec:introduction}
% Introduce the concept of Privacy-Preserving Record Linkage (PPRL) and its importance in sectors such as healthcare, criminal justice, and finance
% Briefly explain the primary challenge of maintaining privacy while linking datasets across different sources
% Introduce the concept of Graph Matching Attacks (GMAs) and the more advanced Dataset Extension Attacks (DEAs)
% State the objective of your thesis: to investigate the vulnerabilities of PPRL systems to DEAs and improve attack techniques using neural networks
Data and record linkage is an important aspect of research and software projects, enabling the integration of data from different sources about the same entity to gain additional insights.
This is particularly important in sectors such as healthcare or social sciences.
In the United States, for example, the fragmented healthcare and public health ecosystem can benefit greatly from effective data linkage.
The COVID-19 pandemic highlighted the critical importance of timely, accurate and efficient data linkage, as the lack of it led to problems in integrating disease and vaccination data.
In response, organisations such as the Centers for Disease Control and Prevention and the Food and Drug Administration have launched projects to address these challenges and further develop linkage techniques. \cite{pathak2024privacy}

In scenarios such as the COVID-19 pandemic, data integration efforts often involve linking records related to natural persons across multiple sources. 
For example, integrating data from various healthcare providers, laboratories, and public health agencies typically requires the use of pseudo-identifiers derived from \ac{pii} such as names, dates of birth, or other sensitive information. 
However, the reliance on \ac{pii} introduces significant privacy concerns, as improper handling of such data can lead to the re-identification of individuals, potentially resulting in severe consequences such as data breaches, identity theft, or unauthorized access to personal health information.

To address these privacy risks, various techniques have been developed to protect \ac{pii}, primarily by encrypting or encoding the data prior to linkage. 
\ac{pprl} techniques are designed to facilitate data integration without exposing sensitive information, ensuring that datasets can be linked securely across different entities. 
However, the use of encrypted \ac{pii} as pseudo-identifiers presents additional challenges. 
Key questions arise regarding how to efficiently encode sensitive information while preserving the ability to match records accurately. 
Furthermore, performing linkage on encrypted or encoded data requires specialized algorithms that can operate effectively without compromising the privacy of the underlying information \cite{}.

In order to still be able to perform linkage while preserving privacy, similarity preserving encodings are applied to the identifiers.
Without such similiraty preserving encoding, matches between encrypted entities in different databases would not be possible.
Over time, three main privacy-preserving encoding schemes have emerged as enablers for \ac{pprl}. \cite{vidanage2020graph, schaefer2024}

%DeppL
\ac{bf} encoding is the most widely used technique in \ac{pprl} and is often regarded as the reference standard. 
\ac{bf}s were introduced to \ac{pprl} due to their simplicity and efficiency in both storage and the computation of private set similarities. 
Their compact representation and probabilistic nature make them ideal for scalable \ac{pprl} systems.
Further research has sought to enhance the security of \ac{bf} encoding by incorporating diffusion layers, which help obscure patterns and reduce vulnerability to certain attacks. % Cite Armknecht
\ac{tmh} is a more recent encoding method that offers distinct advantages in specific use cases. 
While it is less commonly used in \ac{pprl}, its simplicity and efficiency in calculating set similarities make it a viable option. 
Compared to \ac{bf} encoding, \ac{tmh} generally provides stronger security guarantees. 
However, this added security comes at the cost of increased computational complexity and memory usage.
\ac{tsh} introduces a novel approach designed to address some of the limitations associated with \ac{bf} and \ac{tmh}. 
This method employs a two-step process: initially encoding data into multiple \ac{bf}s, followed by an additional hashing step that converts the encoded data into a set of integers for similarity comparisons. 
This layered approach enhances security while maintaining efficient similarity computations. \cite{vidanage2020graph, schaefer2024}

%DeepL
In practice, \ac{bf} based \ac{pprl} has become the dominant standard and is widely adopted in areas such as crime detection, fraud prevention, and national security. 
However, \ac{pprl} systems are not without limitations and vulnerabilities. 
Previous research has demonstrated that several attacks exist targeting \ac{pprl} systems, with a primary focus on exploiting the weaknesses inherent in \ac{bf} encodings.
These attacks specifically target vulnerabilities in \ac{bf} constructions, such as the weaknesses introduced by double hashing, structural flaws in the filter design, and susceptibility to frequent pattern-mining techniques. 
Additionally, language model-based attacks and graph-based dictionary attacks have been employed to compromise \ac{bf} encoded data. 
Notably, no specialized attacks have yet been developed for \ac{tmh} or \ac{tsh} encodings, which suggests that research has primarily concentrated on the more widely used \ac{bf} scheme. \cite{vidanage2020graph}

%DeepL
Nevertheless, a more recent and sophisticated attack has emerged that exploits vulnerabilities common to all \ac{pprl} encoding schemes. 
The \ac{gma} leverages publicly available data, such as phone books, to re-identify encrypted individuals based on overlapping records between plaintext and encrypted databases \cite{vidanage2020graph, schaefer2024}. 
Unlike earlier attacks that focused solely on the encoding scheme of \ac{bf}s, the \ac{gma} operates independently of the encoding scheme by solving a graph isomorphism problem to match records. 
While \ac{gma}s can successfully re-identify individuals present in both the plaintext and encrypted datasets, their effectiveness is limited to the overlapping subset of the two databases \cite{schaefer2024}.

%DeepL
This work aims to advance beyond traditional \ac{gma}s by re-identifying not only individuals present in the overlapping datasets but as many individuals as possible from the encrypted \ac{pprl} data. 
To accomplish this, the newly introduced \ac{dea} builds upon the foundation established by \ac{gma}s. 
The \ac{dea} leverages a neural network trained on the subset of previously re-identified individuals to predict and decode the remaining encrypted records. 
By doing so, the \ac{dea} significantly expands the scope and effectiveness of the attack, enabling broader de-anonymization of \ac{pprl} datasets beyond the limitations of existing graph-based methods.


\section{Motivation}  \label{sec:motivation}
% Discuss the growing importance of data privacy and the increasing need for PPRL systems in sensitive domains
% Highlight the risks associated with PPRL systems when they are vulnerable to attacks such as GMAs and DEAs
% Explain why it is crucial to understand and mitigate these vulnerabilities to ensure the security of linked data
% Mention the lack of sufficient research in extending the scope of attacks beyond the intersection of datasets, which motivates your study
The increasing use of \ac{pprl} in highly sensitive domains such as healthcare, finance, national security, and fraud detection necessitates rigorous research to validate existing techniques and ensure robust data protection. 
As data-driven technologies continue to evolve requiring approaches like \ac{pprl}, so do the methods used by malicious actors to exploit these systems.
While data privacy has always been a critical concern, its importance has surged in an era dominated by \ac{ai} and \ac{ml}. 
These technologies increasingly rely on large datasets, many of which contain \ac{pii} that, if compromised, could lead to significant privacy violations. 
Furthermore, the rise of data brokerage, where personal data is collected, aggregated, and sold—often without explicit user consent has intensified concerns around data misuse. 
As AI models grow more sophisticated, the demand for extensive, high-quality data escalates, making privacy protection a pressing issue \cite{ldc2024,cacgroup2024,arxiv2024}.

In this context, the vulnerabilities of \ac{pprl} systems to emerging attack vectors become particularly alarming. 
As highlighted in the introduction, researchers have demonstrated that \ac{pprl} systems are susceptible to \ac{gma}s, which exploit the similarity preserving properties of common encoding schemes to re-identify individuals. 
These attacks directly undermine the primary objective of \ac{pprl}: to protect sensitive data during the linkage process. 
Although current \ac{gma}s are limited to re-identifying individuals present in both encrypted and plaintext datasets, this still represents a significant privacy risk, especially in domains where even partial data exposure can have serious implications.

The potential implementation of a \ac{dea} introduces even greater risks. 
Unlike \ac{gma}s, which are confined to the intersection of datasets, \ac{dea}s aim to extend the attack to re-identify as many individuals as possible from the encrypted database. 
By leveraging neural networks trained on previously decoded data, \ac{dea}s can predict and decode additional records, potentially leading to the complete de-anonymization of entire encrypted datasets. 
This not only nullifies the privacy guarantees offered by \ac{pprl} systems but also raises critical questions about the future viability of widely used techniques, such as \ac{bf} based \ac{pprl}.

The primary motivation for this research is to proactively investigate and demonstrate the consequences of such advanced attacks to prevent their realization in real-world scenarios.
By exposing the potential vulnerabilities of \ac{pprl} systems, this work aims to highlight how attackers could exploit encrypted data to compromise privacy at scale. 
A successful implementation of the \ac{dea} will reveal that state-of-the-art methods are insufficiently secure, emphasizing the urgent need for more robust privacy-preserving techniques.
Moreover, the lack of comprehensive research into extending the scope of attacks beyond the intersection of datasets underscores the necessity of this study. 
While significant efforts have been made to address the vulnerabilities exposed by \ac{gma}s, there is a notable gap in understanding how these systems can be compromised on a broader scale.
By addressing this gap, this thesis seeks to contribute to the body of knowledge on \ac{pprl} vulnerabilities and serve as a foundation for future research aimed at fortifying these systems against increasingly sophisticated threats.



\section{Related Work}  \label{sec:rel-work}
% Provide a literature review on existing PPRL literatur, encoding methods like Bloom filters, MinHash, and two-step hashing
% Discuss previous research on GMAs and their limitations
% Highlight the gap in current research regarding decoding complete databases.

The work of Vidanage et al. \cite{vidanage2020graph} introduces a novel attack against \ac{pprl}, known as the \ac{gma}.
There the authors first provide an overview about \ac{pprl}, encoding methods and present attacks on \ac{prrl} systems so far.
Their novel attack exploits the similarity preserving properties of commonly used encoding schemes, such as \ac{bf}, making it universally applicable across various \ac{pprl} methods.
By utilizing a graph-based approach and solving a graph-isomorphism problem, the \ac{gma} aligns nodes in similarity graphs to successfully re-identify individuals using a plain database \cite{vidanage2020graph}.
This work is critical for this thesis, as the \ac{dea} builds upon the re-identified individuals produced by the \ac{gma} and is extending the attack.

Another key contribution comes from Schäfer et al. \cite{schaefer2024}, who revisit and extend the work of Vidanage et al. \cite{vidanage2020graph}
The authors provide a thorough reproduction and replication of the proposed \ac{gma}, uncovering an undocumented preprocessing step in the original codebase that unintentionally impacted the attack's success rate.
This undocumented prepropcessing step was intended to improve performance but instead introduced implementation errors.
Schäfer et al. addressed this issue by correcting the preprocessing and enhancing the \ac{gma} to improve both robustness and efficiency.
Their improved implementation achieved higher re-identification rates compared to the original approach by Vidanage et al. \cite{schaefer2024}.
The work of Schäfer et al. is particularly relevant to this thesis, as their enhanced \ac{gma} implementation and corresponding code base serves as the foundation for the \ac{dea} proposed in this thesis.
Therefore small adjustments to the \ac{gma} implementation will be done to build upon it.

Currently there is no approach like the proposed \ac{dea} in research present. Therefore this thesis aims to fill this gap.


%Bloom Filters

%MinHash

%Two Step Encoding

%GMA by Jochen and Vidanage

%Gap for decoding complete database


\section{Contribution}  \label{sec:contribution}
%Summarize the key contributions of your thesis:
% - Comprehensive analysis of the vulnerabilities in PPRL systems using GMA's
% - Enhancement of the existing attack approaches by introducing Dataset Extension Attacks using Neural Networks
% - Development and application of a neural network-based DEA using the provided codebase
% - Evaluation of the impact of different encoding schemes on the effectiveness of DEAs
% - Provide insights into the broader implications of these attacks on data privacy

The contribution of this thesis is structured into three primary components. 
First, a comprehensive analysis of \ac{pprl} systems is conducted, with particular emphasis on the three major encoding schemes: \ac{bf} Encoding, \ac{tsh} Encoding, and \ac{tmh} Encoding. 
This analysis aims to highlight the fundamental principles, strengths, and vulnerabilities of each encoding method, setting the stage for the subsequent investigation of their susceptibility to privacy attacks.

Next, the current state-of-the-art \ac{gma} is analyzed, and its limitations are discussed in detail. 
Although \ac{gma}s have proven effective in re-identifying individuals within overlapping datasets, their applicability is restricted to the intersection of plaintext and encrypted records. 
This inherent limitation underscores the need for more advanced attack strategies that can extend beyond this.

The primary focus of this thesis is the implementation and evaluation of the \ac{dea}, which seeks to surpass the capabilities of \ac{gma}s by decoding a larger proportion of encrypted records. 
To achieve this, the thesis delves into the conceptual foundations, theoretical underpinnings, and technical requirements of the \ac{dea}. 
Building upon the initial re-identifications obtained through the \ac{gma}, the \ac{dea} employs a supervised machine learning approach, specifically utilizing neural networks trained on previously decoded data to predict and re-identify remaining encrypted records. 
This method significantly broadens the scope of de-anonymization in \ac{pprl} systems and provides a novel approach to the current research.

The \ac{dea} is then evaluated against the three major \ac{pprl} encoding schemes. 
While the specific encoding method has minimal impact on the \ac{gma}, which relies primarily on solving graph isomorphism problems, it plays a pivotal role in the \ac{dea}. 
This is due to the fact that the neural network must be trained separately for each encoding scheme to account for the unique structural characteristics and encoding nuances. 
However, the \ac{dea} is designed with adaptability in mind, ensuring that it can be effectively applied across different encoding schemes, thus enhancing its generalizability and practical relevance.

Through this research, the thesis aims to answer critical questions concerning the robustness of \ac{pprl} systems. 
It investigates how effective supervised machine learning-based \ac{dea}s are in re-identifying the remaining entries that \ac{gma}s cannot uncover.
Furthermore, it explores how different encoding schemes influence the performance and accuracy of the \ac{dea}, providing insights into which schemes are more susceptible to such attacks and why. 
By addressing these questions, the thesis contributes to a deeper understanding of the vulnerabilities inherent in \ac{pprl} systems and lays the groundwork for developing more secure privacy-preserving techniques.




\section{Organization of this Thesis}  \label{sec:orga}
% Outline the structure of the tesis:
% - provide an overview of PPRL systems and common encoding techniques
% - review existing work on GMAs, setting the stage for my contribution
% - detail the methodology used in my study, including the neural network implementation
% - Do the actual work and implement something
% - present the results of the DEA across different encoding schemes and discusses the findings
% - conclude the thesis with a summary of contributions, implications, and suggestions for future research

This thesis is divided into X main sections.
First, an overview of \ac{pprl} systems will be provided, with a particular focus on a thorough analysis of the most commonly used encoding techniques.
Following this, the existing \ac{gma} will be introduced and explained to establish the foundation for the study.
Additionally, an overview of neural networks will be presented to provide necessary background knowledge.

Next, a detailed description of the attack model for the \ac{dea} will be outlined, including how neural networks are leveraged to enhance the attack.
This is followed by an explanation of the actual implementation of the \ac{dea}, along with a discussion of the experiments conducted.
The results of the \ac{dea} across different encoding schemes will then be analyzed and evaluated.
Finally, the thesis will conclude with a summary of the key contributions, a discussion of the broader implications, and suggestions for future research.