% !TeX root = main.tex
%%%%%%%%%% SVN Info %%%%%%%%%%%%%%%%%%%%%%%%%%%%%%%%%%%%%%%%
%% $Date: 2024-07-23 16:35:07 +0200 (Tue, 23 Jul 2024) $
%% $Author: cmueller $
%% $Rev: 79 $
%%%%%%%%%%%%%%%%%%%%%%%%%%%%%%%%%%%%%%%%%%%%%%%%%%%%%%%%%%%%

\listfiles    % Prints all required files to the log

%%%%%%%%%%  Packages & Options  %%%%%%%%%%
\usepackage{blindtext}
\usepackage{xifthen}            % Offers some more sophisticated if-then-else stuff
\usepackage{ifdraft}            % Provides simple draft/final mode if-then
\usepackage{ifluatex}           % Detects Lua(la)tex engine, provides if-then-else command
\usepackage[%
    draft=false,                % Remove rulers in draft mode
    , headsepline               % Show header separation line
%    , footsepline               % Show footer separation line
    ]{scrlayer-scrpage}         % The recommended package for headers and footers
\usepackage{amsmath}            % Some maths stuff
\usepackage{amsthm}             % Some more maths stuff
\usepackage{amssymb}            % And even more maths stuff
\usepackage{siunitx}            % For typesetting SI units nicely
\usepackage{csquotes}           % Advanced quotation stuff, recommended for biblatex
\usepackage{booktabs}           % For high-quality tables
\usepackage{multirow}           % To have nicer table layout (merging 2 rows in tables)
\usepackage{multicol}           % Also for nicer table layout (but columns this time)
%%
\ifluatex                       % If compiling with LuaTeX
  \usepackage{fontspec}           % Requires XeTeX or LuaTeX, provides easier font management
  \usepackage{polyglossia}        % Modern hyphenation and more language-dependent settings, recommended for Lua/fontspec
  \setdefaultlanguage[%
    variant = american]%          % Set american...
    {english}                     % ...english for polyglossia
  \setotherlanguage{german}       % Also load the german language files
  \setsansfont[%
    Scale = .92]%                % Use scaling factor (92%) to match normal font size
    {TeXGyreHeros}              % Use Helvetica-replacement as sans serif font
  \usepackage[%                   % Preferred settings
    sortcites = true,               % Sort references
    style     = alphabetic,         % Use alphabetic style
    maxnames  = 15,                 % Only use "et al." in for more than 15 authors (in general)
    maxalphanames = 4,              % Use max 4 authors last names for labels in alphabetic style
    minalphanames = 3,              % Use 3 author names in labels if more than maxalphanames authors present, generates sth. like [ABC+12]
    backend   = biber]              % Explicitly set to use biber
    {biblatex}                    % BibLaTeX
\else                           % No LuaTeX detected
  \usepackage[T1]{fontenc}        % Font encoding
  \usepackage{lmodern}            % Modern fonts
  \usepackage[scaled=.92]%        % Scale to 92% to match Times 12pt (defaults to 95%)
    {helvet}                      % Use Helvetica as sans serif font
  \usepackage[ngerman             % Use 'new' german language as secondary...
    , main=american               % ... use american english as main language
    ]{babel}                      % Hyphenation and more language-dependent settings
    \babeltags{german=ngerman}    % Define environment for German text, similar to polyglossia
  \usepackage[%                   % Assuming no LuaLaTeX also means no Biber, so use bibtex as fallback
    sortcites = true,               % Sort references
    style     = alphabetic,         % Use alphabetic style
    maxnames  = 15,                 % Only use "et al." in for more than 15 authors (in general)
    maxalphanames = 4,              % Use max 4 authors last names for labels in alphabetic style
    minalphanames = 3,              % Use 3 author names in labels if more than maxalphanames authors present, generates sth. like [ABC+12]
    backend = bibtex8]              % Choose backend for bib processing, biber recommended but bibtex[8] driver also works (limited)
    {biblatex}                    % BibLaTeX
\fi
%%
\usepackage[%
    useregional                 % Use regional date style
]{datetime2}                  % Obvious, should be loaded after babel/polyglossia

\usepackage[%
    dvipsnames,                 % Load dvips driver colors immediately
    x11names,                   % Load x11names colors immediately
    cmyk]                       % Convert all colors to CMYK model
    {xcolor}                    % Obvious, but load before TikZ
\usepackage{tikz}               % For fancy graphics and so on
\usepackage[%
    xcolor                      % Use the xcolor package, already loaded
    , framemethod = tikz        % Use tikz for framing options
    ]{mdframed}                 % For drawing (colored) boxes around amsthm stuff
\usepackage[%
    obeyFinal                   % Disable notes in final state
%   , obeyDraft                 % Enable notes only in draft state
    ]%
    {todonotes}                 % Enables fancy todo sticky-notes in page margin
\usepackage{listings}           % For typesetting source code in a nice way
\usepackage{algorithm2e}        % For typesetting algorithms in pseudocode
\usepackage{acro}               % Provides stuff for acronyms
\usepackage[%
    colorlinks= true,           % Use different color for links, colors are defined later
    breaklinks= true,           % Allow linebreaks in links
    bookmarks = true,           % Create bookmarks from sections etc.
    linktoc = all               % Make Section names and page numbers links in ToC
  ] {hyperref}                  % Make clickable hyperlinks
\usepackage[listings]{scrhack}  % Correct listings macros to work with KOMA-Script
\usepackage{hypcap}             % Help hyperlinks to point to 'more correct' location
%%%% IMPORTANT: cleveref needs to be loaded last %%%%
\usepackage[%
    capitalize,                 % Capitalize all identifiers
    noabbrev]                   % Do not abbreviate
    {cleveref}                  % For sophisticated/automated referencing
\usepackage{float}

%%%%%%%%%% More Settings %%%%%%%%%%
%%%% Define theorem env's %%%%
\newtheoremstyle{myStyle}  % name
  {\topsep}                % space before
  {\topsep}                % space after
  {}                       % body font (was: \itshape)
  {}                       % indentation
  {\bfseries\sffamily}     % header font
  {.}                      % punctuation
  {.5em}                   % space after header
  {}                       % head spec, defaults to 'normal'
\theoremstyle{myStyle}

\mdfdefinestyle{myMdfStyle}{%
  innertopmargin = 0pt,
  innerbottommargin = 10pt,
  innermargin = 0pt,
  outermargin = 7pt,
  linewidth = 1pt,
  linecolor = black,
  backgroundcolor = lightgray!30,
  roundcorner = 5pt%
}

\newmdtheoremenv[%
  style = myMdfStyle%
]{definition}{Definition}[section]

%%%% Load required TikZ libs %%%%
\usetikzlibrary{chains, calc, shapes}

%%%% Colors %%%%
\definecolor{notes-color}{cmyk}{.51, .23, 0, 0}
\definecolor{my-color}{cmyk}{0.5, 0.0, 0.5, 0.2}
\definecolor{uma-main}{cmyk}{1, 0.9, 0.35, 0.25}
\definecolor{uma-wim}{cmyk}{0.53, 0.0, 0.02, 0.46}

%%%% Hyperref coloring %%%%
\hypersetup{%
  citecolor = uma-main,       % Use UMA-Blue for citations
  linkcolor = uma-main,       % Use UMA-Blue for links
  urlcolor  = magenta,        % Use standard magenta for URLs
  anchorcolor = black         % Use standard black for anchors
}

%%%% Acronym setup %%%%
\acsetup{
  single = true,              % Do not make used-only-once acronyms real acronyms
  single-style = long,        % Use the long form for used-only-once acronyms
  make-links = true,          % Provide a link to the definition in the list
  list/sort = true,           % Try sorting the acronyms alphabetically by short form
%  format/short = {\upshape\bfseries\sffamily},  % Use sans-serif+bold+up for short-form list entry
}

%%%% Listings setup %%%%
\newcommand{\clstrefname}{Code Snippet}
\newcommand{\clstrefnames}{Code Snippets}

\lstset{numberbychapter = true        % Listings are numbered by chapter and not (globally) sequentially
}

\lstset{
  showstringspaces=false,
  basicstyle=\normalsize\mdseries\ttfamily,
  keywordstyle=\color{blue},
  commentstyle={\ttfamily\itshape\color{Chartreuse4}},
  stringstyle=\mdseries\color[cmyk]{0, .41, .71, .40 },
  breaklines=true,
  backgroundcolor=\color{Snow2},
  xleftmargin=1em,
  xrightmargin=0em,
  numbers=left,
  numberstyle=\tiny,
  stepnumber=1,
  numbersep=5pt,
  tabsize=2,
  literate={°}{\textdegree}1,
  language=java
}

\newcommand{\inlinecode}[2][java]{\colorbox{Snow2}{\lstinline[language=#1]$#2$}}

\renewcommand{\lstlistingname}{\clstrefname}
\renewcommand{\lstlistlistingname}{List of \clstrefnames}


%%%% Note setup %%%%
\presetkeys{todonotes}{color=notes-color, size=\small}{}
\newcommand{\myNote}[1]{\todo[color={my-color}, size=\tiny]{\textsf{\textbf{[Note:]}} #1}}

%%%% Define cross-referencing %%%%
%\crefname{definition}{Definition}{Definitions}
%\Crefname{definition}{Definition}{Definitions}
%\crefname{equation}{Equation}{Equations}
%\Crefname{equation}{Equation}{Equations}
%\crefname{algorithm}{Algorithm}{Algorithms}
%\Crefname{algorithm}{Algorithm}{Algorithms}
%%% Hack to rename Listings to \clstrefname as the high-level command doesn't work
%% Standard reference formats
\crefformat{listing}{\clstrefname~#2#1#3}
\crefrangeformat{listing}{\clstrefnames~#3#1#4 to~#5#2#6}
\crefmultiformat{listing}{\clstrefnames~#2#1#3}%
  { and~#2#1#3}{, #2#1#3}{ and~#2#1#3}
\crefrangemultiformat{listing}{\clstrefnames~#3#1#4 to~#5#2#6}%
  { and~#3#1#4 to~#5#2#6}{, #3#1#4 to~#5#2#6}{ and~#3#1#4 to~#5#2#6}
%% Beginning-of-Sentence reference formats
\Crefformat{listing}{\clstrefname~#2#1#3}
\Crefrangeformat{listing}{\clstrefnames~#3#1#4 to~#5#2#6}
\Crefmultiformat{listing}{\clstrefnames~#2#1#3}%
{ and~#2#1#3}{, #2#1#3}{ and~#2#1#3}
\Crefrangemultiformat{listing}{\clstrefnames~#3#1#4 to~#5#2#6}%
{ and~#3#1#4 to~#5#2#6}{, #3#1#4 to~#5#2#6}{ and~#3#1#4 to~#5#2#6}
%%%

%%%% Change some labelling typesetting
\setkomafont{captionlabel}{\upshape\bfseries\sffamily}

%%%% Set the algorithm layout
\RestyleAlgo{algoruled}
\LinesNumbered
\SetNlSty{textsf}{}{}
%\hypcapredef{algorithm}
%% Style the Algorithm caption accordingly
\renewcommand\AlCapFnt{\sffamily}

%%%% Set the header and footer
\pagestyle{scrheadings}

%%%% Numbering subsubsections
\setcounter{secnumdepth}{\subsubsectionnumdepth}

%%%% Include subsubsections in ToC
\setcounter{tocdepth}{\subsubsectiontocdepth}

\recalctypearea   %% Recalculate type area after fonts and everything is set
